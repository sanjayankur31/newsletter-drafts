\documentclass[12pt,a4paper]{article}
\usepackage[default,osfigures,scale=0.95]{opensans} % for e-copy
\usepackage[pdftex,hyperfootnotes=false,pdfpagelabels]{hyperref}
\usepackage{graphicx}
\title{NEST, the Neural Simulation Tool}
\author{Ankur Sinha}
\begin{document}
\maketitle
\begin{figure}[h]
  \centering
  \includegraphics[width=0.4\textwidth,keepaspectratio]{nest-initiative.png}
\end{figure}
% Introduction
The \href{http://www.nest-simulator.org}{NEST (the Neural Simulation Tool)} is one of the many software tools that are used for modelling of brain mechanisms.
Over \href{http://www.nest-simulator.org/publications/}{300 models} have now been implemented in NEST since its initial release in 1994 as SYNOD\@.
Since then, NEST has grown via contributions from developers from the world over, under the guidance of the core team---\href{http://www.nest-initiative.org/membership/}{the NEST Initiative}.
The most recent release, \href{https://github.com/nest/nest-simulator/releases/tag/v2.14.0 }{version 2.14.0}, saw 700 contributions from as many as 30 developers---many of them PhD candidates.
With a Python API that enables quick prototyping, an active community of users and maintainers, and an \href{https://en.wikipedia.org/wiki/Open-source_model}{Open source} development model that encourages collaboration while maintaining software standards, NEST is now considered one of the go-to simulators for modelling spiking neuronal networks.

% Ease of use
NEST provides a high level API in the Python programming language that presents a simple, unified interface for researchers with different levels of technical expertise to build their models on.
The API exposes the more than 50 built-in neuron and many synapse models to the user.
It also includes efficient methods to create networks and helper functions to modify and inspect the states of simulation entities.
Detailed \href{http://nest-simulator.org/documentation/}{documentation and example scripts} further supplement the API\@.
In cases where these are insufficient, an active \href{http://mail.nest-initiative.org/cgi-bin/mailman/listinfo/nest_user}{mailing list} allows users to communicate with the NEST community.
As an added feature, the simulator-independent \href{http://neuralensemble.org/PyNN/}{PyNN} framework also supports NEST\@.

% Scalability
NEST is designed to take advantage of the increasingly parallel hardware that is available nowadays.
The same simulation code can be run on different hardware configurations without requiring the user to make any changes at all---from multicore PCs to clusters of connected computing cores.
The distribution of computing tasks over the available hardware resources is handled internally by NEST\@.
Users need not have any knowledge of parallel programming.
Indeed, this has enabled researchers to use NEST to simulate thousands of neurons and millions of synapses to \href{https://academic.oup.com/cercor/article/24/3/785/398560}{study near full-scale models of local cortical microcircuits} using computing clusters.
The development team is in the process of making further optimisations to improve performance and extend usability---these will be all part of the next NEST release---NEST 3.0.

% Community based development
As a Free/Open source software project, \href{https://github.com/nest/nest-simulator}{NEST is developed openly on Github}.
Users can engage in the development process here---file bugs, discuss and test new features, provide feedback on changes, and even make additions to the simulator themselves.
The development team also holds an open \href{https://github.com/nest/nest-simulator/wiki/Open-NEST-Developer-Video-Conference }{fortnightly NEST development video conference} to provide developers a channel for face to face discussion.
Not only does the development model keep users aware of changes, it provides a great platform for students and early researchers to get involved with the larger neuroscience community.
It also helps researchers understand the underlying work that goes into the creation of a high performance computational modelling tool.


% Miscellaneous information
The increasing stress on computational modelling to better understand neural mechanisms make software such as NEST important tools in a researcher's armoury.
The steadily increasing user and contributor base suggest that NEST will continue to remain a key simulator in neuroscience research.
NEST can be downloaded from \href{http://nest-simulator.org/download/}{nest-simulator.org}.
Documentation, examples, and links to other resources such as the mailing list can also be found there.

\end{document}
