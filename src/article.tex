\documentclass[12pt,a4paper]{article}
\usepackage[default,osfigures,scale=0.95]{opensans} % for e-copy
\usepackage[pdftex,hyperfootnotes=false,pdfpagelabels]{hyperref}
\title{NEST, the Neural Simulation Tool}
\author{Ankur Sinha}
\begin{document}
\maketitle
Among the various simulators that computational neuroscientists use for modelling the brain is the \href{http://www.nest-simulator.org}{NEST (Neural Simulation Tool)}. Over \href{http://www.nest-simulator.org/publications/}{300 models} have been implemented in NEST since its initial release in 1994 as SYNOD\@. Since then, NEST development has thrived via contributions from developers from the world over under the watchful eye of \href{http://www.nest-initiative.org/membership/}{the NEST Initiative}. The recent \href{https://github.com/nest/nest-simulator/releases/tag/v2.14.0 }{NEST 2.14.0 release}, for example, saw contributions from about 50 developers. With a Python API that enables quick prototyping, an active community of users and maintainers, and an \href{https://en.wikipedia.org/wiki/Open-source_model}{Open source development model}, NEST is now considered one of the go-to simulators for modelling spiking neuronal networks, especially ones that feature large numbers of neurons.

NEST is aimed at the modelling of neuronal networks that use spiking neuron models on parallel hardware such as multi-core computers and clusters. This enables rather large scale simulations

\end{document}
